\documentclass[20pt,a4paper]{report}
\usepackage[french]{babel}
\usepackage[top=3cm,bottom=3cm,left=2.5cm,right=2.5cm]{geometry}
\usepackage{graphicx}
\usepackage{caption}
\title{ Rapport de Stage \\ Theme : }
\author{FERHAT Rania}
\date{}
\usepackage{hyperref}
\usepackage{fancyhdr}
\pagestyle{fancy}

\renewcommand{\headrulewidth}{1pt}
\fancyhead[L]{\leftmark}
\fancyhead[R]{ENPO}

\renewcommand{\footrulewidth}{1pt}
\fancyfoot[C]{\textbf{ \thepage}} 
%\fancyfoot[L]{ENPO}
%\fancyfoot[R]{\leftmark}


\begin{document}
\begin{titlepage}
\begin{center}
\large{\textbf{République Algérienne Démocratique et Populaire \\
Ministère de l’Enseignement Supérieur et de la Recherche Scientifique \\}}

\vspace{1cm}
Ecole Nationale Polytechnique d'Oran -Maurice Audin- \\

\includegraphics[scale=0.5]{../../SPECIALITE/logo-enpo.png} 

Département génie mécanique

2ème année Systèmes Enérgétiques
\vspace{2cm}
\hrule
\vspace{1cm}
\LARGE{\textbf{Rapport de Stage\\ Theme}}
\vspace{1cm}
\hrule
\vspace{2cm}
éffectué au sein de \\

\includegraphics[scale=0.5]{../../logo-removebg-preview.png} 

du ..... au .... 2023

\vspace{3cm}
Effectué par : FERHAT Rania
\end{center}
\end{titlepage}
	
\begin{center}
\textbf{{\LARGE Remerciement}}
\end{center}


\begin{center}
\begin{LARGE}
Je tiens à exprimer ma profonde gratitude et mes sincères remerciements à l’Entreprise Nationale Algérienne de Peinture d’avoir agrée ma demande de stage.
\\
A l’équipe du service ressources humaines de nous avoir accueilli.
\\
Aux techniciens d’avoir pris le temps pour nous apprendre et avoir répondu à toutes nos questions, grâce à eux ce stage fut intéressant et valorisant.
\\

Aux agents pour leur gentillesse et leur sympathie.
\end{LARGE}
\end{center}

	\newpage

	
	\tableofcontents
	
	\newpage	
	
	\listoffigures
	
	\newpage
	
	\chapter{Introduction}
	
	\newpage
	
	\begin{large}
	Ce rapport est l’aboutissement de deux semaines de stage d’insertion au sein de la Société de Production de l’Électricité effectué du 25 mars au 8 avril 2023 dans le cadre de ma scolarité à l’École Nationale Polytechnique d’Oran (stage de 2ème année cycle ingénieur en mécanique option systèmes énergétiques).

	Le secteur de fabrication des peintures est un secteur de taille moyenne dans le domaine de la parachimie, il est indispensable car la peinture est toujours utilisée dans de diverses activités industrielles.
	\end{large}
	
		\begin{large}
		\section{Objectifs}
		\begin{itemize}
			\item Découvrir le milieu professionnel.
			\item Confronter des connaissances théoriques à des cas pratiques.
			\item Comprendre le processus de production de peinture.
			\item L’utilité des systèmes d’échange de chaleur dans une entreprise.
		\end{itemize}
		\end{large}
		
		\begin{large}
		\section{Présentation de l’entreprise}

L’ENAP est une entreprise spécialisée en domaine de la production de la peinture comprend six (6) unités de production réparties sur le territoire national dont une unité implantée dans la zone industrielle de Sig dans la wilaya de Mascara qui produit 125.000 tonnes de peintures et et  57.000 tonnes de produits semi-ouvrés qui s’étale sur un site d’une superficie de  13 hectares.

Son objectif est la fabrication et  de diminuer l’importation des peinturee et de satisfaire la demande nationale la commercialisation de 
		\begin{itemize}
		\item peintures et vernis
		\item peintures de batiments
		\item printure industrie
		\item
		\end{itemize}




		\section{Historique}

Suite au manque de   de peinture prévu à partir de 1980, le gouvernement a mis en point un programme de construction de 2 complexes de peinture d’une capacité de 40000 tonnes chacun, l’un à Sig et l’autre à Souk Ahras.

1974 : le lancement des appels d’offre.

14 juin 1975 : la signature d’un contrat avec une société (BECSE).

Juin 1977 : commencement des travaux de terrassement V.R.D par la société algérienne SONATRO 

Janvier 1984 : le début de la production en la partageant en 2 unités autonomes, une chargée de la distribution et l’autre de la production.
		\section{Organigramme} 
 l'organigramme est comme ceci
		\end{large}
 			
 			\begin{figure}[h]
 			\centering
 			\includegraphics[scale=0.5]{../../SPECIALITE/stage/4/organigramme.png}
 			\end{figure}
 			
 \newpage
		\begin{large}
		\section{La maintenance}
La maintenance est un élément essentiel pour le bon fonctionnement et la durabilité des équipements d’une entreprise, elle consiste à maintenir, réparer et remplacer les équipements afin de garantir leur disponibilité et efficacité. La maintenance peut être préventive ou corrective.

La maintenance préventive : effectuée avant qu’un équipement ne tombe en panne ou nécessite une réparation. Elle consiste à effectuer des inspections régulières, des remplacements planifiés de pièces et des actions de nettoyage pour maintenir l’équipement en bon état de fonctionnement.

La maintenance corrective : effectuée après qu’un équipement a subi une panne. Elle consiste à identifier la cause de la panne, à remplacer les pièces défectueuses et à réparer l’équipement pour le mettre en état de fonctionnement.
Elle a pour but :

			\begin{itemize}
			\item Assurer la sécurité des travailleurs
			\item Maintenir la qualité de production
			\item Minimiser les couts de réparation et de remplacement
			\item La protection de l’environnement
			\end{itemize}

	\newpage
	
	\chapter{Processus de production}
	
	\newpage

		\section{Présentation des produits}
La peinture est un art aux variantes multiples qui a toujours existé, un matériau fluide ou semi fluide composé de liants, pigments et adjuvants, donnant par application en couches minces ou épaisses sur des subjectiles appropriés, formant .. possédant des propriétés protectrice décoratives ou techniques particulières 
Les pigments : amènent la couleur et l’opacité.

Le liant : maintient ensemble les pigments et forme un film résistant qui adhère à la surface.

Les liquides : donnent la viscosité nécessaire.

Les additifs : des ingrédients aux propriétés spécifiques.

		\section{Processus }
Le processus de production de peinture est un processus complexe qui implique plusieurs étapes pour créer un produit final de haute qualité. Les étapes comprennent :
			\subsection{Formulation : }
Qui est la première étape du processus, elle implique la sélection des matières premières nécessaires pour créer le type de peinture souhaité et sa finesse, les résines, les pigments, les solvants et les additifs.

			\subsection{Pesage : }
Après la formulation, les matières premières doivent être pesées avec précision pour créer la formule de peinture exacte. Des systèmes de pesage sont utilisés.

			\subsection{Prémélange : }
Les matières sont mélangées dans une machine appelée *melangeur*. Cette étape est essentielle pour mélanger, dépendant du produit, et éliminer les agglomérats.

			\subsection{Broyage : }
La peinture est ensuite envoyée dans un broyeur où les pigments sont broyés en particules très fines pour créer une dispersion homogène dans le mélange en supprimant aussi les grains dans la peinture. Cette étape assure la couleur uniforme et la bonne couverture.

			\subsection{Filtration : }
La peinture passe par un processus de filtration pour éliminer les particules indésirables et les impuretés.

			\subsection{Conditionnement : }
Une fois la peinture filtrée, elle est prête à être conditionner et emballée pour la vente. Cette étape implique le remplissage de la peinture dans ses contenant appropriés en l’étiquetage et le conditionnement pour le transport et la distribution.

\newpage

	\chapter{Equipements}
	
	\newpage

La structure de l’usine comprend plusieurs ensembles distincts qui servent à accomplir diverses tâches que nous avons eu l’opportunité de voir, notamment :
		\section{Les pompes}
Les pompes sont des turbomachines qui ont pour caractéristiques de prélever, en un temps donné, un volume de liquide incompressible à l'aspiration, et de l'envoyer au refoulement.

			\subsection{Pompes volumétriques }
Les pompes volumétriques sont un type de pompes qui utilise des volumes fixes de liquide pour générer un débit constant de liquide. On distingue durant notre stage deux technologies correspondant à ces caractéristiques, notamment :

\begin{itemize}
	\item Les pompes à membranes
	\item Les pompes à engrenages
\end{itemize}

				\subsubsection{Les pompes à membranes }
Sont largement utilisées dans les industrie pour le transfert de liquides, notamment dans les secteurs de la chimie, petrochimie..etc

Leur fonctionnement se base sur le déplacement d’une membrane souple qui aspire et refoule le liquide à travers une série de vannes. La membrane crée une pression à l’intérieur de la chambre de pompage, qui pousse le liquide à travers la pompe. Elle peut être utilisée pour pomper une large gamme de liquides visqueux aux liquides corrosifs grâce à sa conception simple et robuste.

Elles sont alimentées en air comprime et utilisées pou le transport de             à l’intérieur des bâtiments.
 
					\begin{figure}[h]
					\centering
					\includegraphics[scale=1]{../../SPECIALITE/stage/4/Schéma d'une pompe à membrane.jpg}
					\caption{Schéma d'une pompe à membrane}
					\end{figure}
					
					\begin{figure}[!h]
					\centering
					\includegraphics[scale=1]{../../SPECIALITE/stage/4/pompe membrane.jpg} 		
					\caption{pompe à membrane}
					\end{figure}
										 
\textbf{Avantages :}	
\begin{itemize}
\item Capables de fonctionner de manière continue sans interruption pendant de longues périodes.

	\item Peuvent être utilisées pour pomper des liquides corrosifs ou dangereux sans risque de fuite.
	
	\item Faciles à démonter et à réparer en cas de besoin.
	
	\item Nécessitent une puissance relativement faible pour fonctionner et ont par conséquence un bon rendement.
\end{itemize}

\textbf{Inconvénients }:
\begin{itemize}
	\item Chères à l’achat.
	
	\item Nécessitent de remplacer régulièrement les membranes, qui peuvent se dégrader avec le temps.
	
	\item Peuvent avoir des débits limités.
\end{itemize}

				\subsubsection{Pompes à engrenages }
Un type de pompe volumétrique qui fonctionne en utilisant deux engrenages pour pomper le fluide. Les engrenages sont souvent des roues dentées, une entrainant l’autre en tournant, ce qui crée une pression pour propulser le fluide.

Elles sont particulièrement adaptées pour le transfert de fluides visqueux sans subir de dommages ou d’usure excessive elles sont alors fiables pour le transport des liquides a hautes viscosités

Son fonctionnement : lorsque les engrenages sont en rotation, le fluide est aspiré dans l’aspiration entre les dents des deux engrenages, ensuite, le fluide est transporté entre les dents et est poussé hors de la pompe par la force créée par la rotation des engrenages. Le débit dépend de la taille des engrenages, leur vitesse de rotation et leur disposition dans la pompe. Elles sont souvent utilisées pour des applications de faible a moyennes pressions
Utilisées pour le transport de l’huile de graissage

					\begin{figure}[h]
					\caption{Schéma d'une pompe à engrenages}
					\centering
					\includegraphics[scale=1]{../../SPECIALITE/stage/4/engrenage.png}
					\end{figure}
					
					
 
\textbf{Avantages} :

\begin{itemize}
	\item Faible cout d’achat.
	
	\item Fonctionnement à des pressions/températures élevées.
\end{itemize}
\textbf{Inconvénients} :

\begin{itemize}
	\item S’usent rapidement.
	
	\item La lubrification peut entrainer une contamination du liquide pompé.
	
	\item Peuvent être bruyantes en fonctionnement.
\end{itemize}

			\subsection{Pompes centrifuges}
Les pompes centrifuges sont un type de pompe qui fonctionne en utilisant une roue à aubes ou une turbine tournante pour transférer de l’Energie cinétique au fluide, ce qui crée une pression qui permet de pomper le liquide.

 Elles fonctionnent suivant le principe d'une mise en rotation du fluide à pomper dans une roue tournant à grande vitesse (~600 - 3500 tr.mn¯¹). En sortie de roue, le fluide est canalisé dans un diffuseur, puis ralenti dans une volute, et la pression dynamique acquise au niveau de la roue (énergie de vitesse ou cinétique) est transformée en pression statique (énergie de pression). Le débit pompé est essentiellement fonction de la différence de pression entre aspiration et refoulement (en m CL) et de la vitesse de rotation de la roue
 
\textbf{Son fonctionnement} : lorsque le fluide pénètre dans la pompe à travers l’aspiration, il est dirigé vers le centre de la roue à aubes ou de la turbine. Sa rotation transfère de l’énergie cinétique au fluide ce qui augmente la pression et propulse le fluide hors de la pompe par la sortie.

Elles sont utilisées dans des applications où une pression élevée n’est pas nécessaire mais un débit important l’est, également où le liquide à pomper contient des particules solides ou des fibres, car elles ont une ouverture d’aspiration plus large et sont moins susceptibles de se boucher que d’autres types de pompes.
Elles sont alimentées en électricité utilisées pour le transport de l’eau de refroidissement des réservoirs au sous-sol aux tours.
       
 \begin{figure}[!h]
 \centering
 \includegraphics[scale=1]{../../SPECIALITE/stage/4/schema centrifu.PNG}
 \caption{Schéma d'une pompe centrifuge
 }
 \end{figure}
 
 
					\begin{figure}[!h]
					\caption{pompes centrifuge}
					\centering
					\includegraphics[scale=1]{../../SPECIALITE/stage/4/centrifuge.jpg}
					\end{figure}
					
				\textbf{Avantages} :
	Capables de transférer des liquides à haute vitesse.
	
	Faible vibration et faible bruit.
	
	Leur conception simple facilitant l’installation et la maintenance.
	
	Fonctionnent à sec pendant une courte période sans être endommagées.
\textbf{Inconvénients} :
	Leur faible pression de refoulement, limitant leur utilisation.
	
	Sensibles aux changements de viscosité et de densité du liquide.
	
	Ont une durée de vie plus courte que les pompes alternatives en raison de la vitesse élevée à laquelle elles fonctionnent.
	
	Nécessitent des dispositifs supplémentaires pour le contrôle de la température et la pression.
\\
\\
\\
\\
		\section{Les compresseurs}
Un compresseur à vis qui est un type de compresseur qui utilise des vis en rotation pour comprimer l’air ambiant. Les vis sont conçues pour se déplacer l’une par rapport à l’autre tout en créant un espace de compression qui réduit progressivement la taille de la chambre de compression.

Couramment utilisées dans les applications industrielles pour fournir de l’air comprimé pour l’alimentation en air pour les outils pneumatiques, les systèmes de contrôle et de mesure, les systèmes de climatisation ..etc

L’entreprise est équipée d’un compresseur KAESER KOMPRESSOREN avec film d’huile est utilisé pour lubrifier les vis réduisant les frottements et l’usure, cependant cela peut poser des problèmes de contamination de l’air comprimé, c’est pourquoi un filtre à huile est inclus dans l’ensemble.

L’air ambiant est aspiré filtrés et dirigé vers un compresseur à vis avec un film d’huile ensuite un séparateur qui sépare l’air de l’huile, un radiateur pour adapter sa température à celle souhaitée et stocké dans les ballons d’air avec un manomètre pour contrôler sa pression et des soupapes de sécurité de surpression.

Il est commandé électroniquement avec le KAESER SIGMA CONTROL affichant sa pression de consigne, la température de l’air ..etc

leur fonctionnement

\begin{figure}[h]
\centering
\includegraphics[scale=1]{../../SPECIALITE/stage/4/schema compresseur.jpg}
\caption{Schéma d'un compresseur Kaeser}
\end{figure}

 
Un autre type de compresseur, CompAir L132 RS est utilisé d’une puissance de 132 KW, d’un débit d’air variant de 18.7 à 24.79 m³/mn pour des pressions de cosigne de 7.5 à 13 bar niveau sonore 76 dB contrôlé par un contrôleur à écran tactile innovant Delcos XL.

\begin{figure}[h]
\centering
\includegraphics[scale=1]{../../SPECIALITE/stage/4/compair.jpg}
\caption{Compresseur CompAir
}
\end{figure}


 
Ils sont utilisés conformément aux mesures de sécurité qui exigent l’alimentation de plusieurs équipements par l’air comprimé fonctionnent avec de vérins pneumatiques au lieu de l’électricité.

\textbf{Avantages} :

\begin{itemize}
	\item Efficaces et ont un rendement élevé.
	
	\item Nécessitent peu d’entretien en les comparant avec d’autre types de compresseurs.
	
	\item Compacts et prennent peu de place.
	
	\item Ont une plage de de vitesse très large, leur permettant de s’adapter facilement aux variations de la demande en air comprimé.
\end{itemize}

\textbf{Inconvénients} :

\begin{itemize}
	\item Plus chers à l’achat.
	
	\item Peuvent générer des vibrations et du bruit.
	
	\item La qualité de l’air comprimé par ce genre de compresseurs peut être inférieure à celle produite par d’autre
	Ont tendance à surchauffer ce qui peut nécessiter un refroidissement supplémentaire.
\end{itemize}

		\section{Sécheur }
Est une machine utilisant l’air et une source d’énergie pour sécher l’air de tout liquide former lors de la compression de l’air

Un équipement utilisé pour éliminer l’humidité de l’air comprimé utilisé dans les process industriels. L’air ambiant, qui contient généralement de l’humidité, ou l’humidité qui se forme après la compression peut causer des problèmes dans les applications industrielles telles que la corrosion, la formation de givre, la détérioration des équipements pneumatiques et la contamination des produits.

Il existe plusieurs technologies mais les constructeurs ont opté pour un type de sécheur par réfrigération en utilisant un ventilateur qui consiste à refroidir l’air comprimé en le faisant circuler à travers un échangeur de chaleur qui abaisse de sa température pour condenser l’humidité.

		\section{Le réacteur}
C’est un dispositif dont à l’intérieur de lui se passe une réaction d’un mélange , ayant le but de faire une réaction entre une poudre chimique spéciale et l’huile de soja pour produire de la matière semi finie
Il permet de mélanger et de réagir différents ingrédients pour produire une formule de peinture spécifique. 

Le réacteur peut être équipé de différents types d’agitateurs pour mélanger efficacement les ingrédients  
Ils peuvent également être conçus pour maintenir une température et une pression spécifique pour optimiser les réactions chimiques



		\section{Le broyeur}
Un équipements utilisé dans de nombreuses industries pour réduire la taille des matériaux en particules plus petites, il peut être utilise pour broyer une variété de matériaux tels que le bois, le plastique, le verre, les produits chimiques..etc

Son fonctionnement : peuvent être conçus pour fonctionner de différentes manières de différents mécanismes de broyage tel que le cisaillement, le concassage, meulage.

Les broyeurs peuvent être utilisés pour différentes applications telles que le recyclage, la production d'énergie, la production de matériaux pour la construction, l'agriculture et bien plus encore. Les broyeurs peuvent être utilisés pour créer des particules plus petites qui sont plus faciles à manipuler, à stocker et à transporter.

		\section{La conditionneuse }
Un équipement utilisé dans l’industrie pour emballer, conditionner et étiqueter différents types de produits, équipée de mécanismes permettant le remplissage et la fermeture pour remplir les produits dans des contenants tels que les sachets des boites ou des pots. Elle a l’avantage d’augmenter l’efficacité de la production en emballant a une vitesse élevée.

Elles fonctionnent avec des vérins pneumatique double effet et de l’air comprimé comme source d’énergie.

\newpage


	\chapter{Systèmes d’échange de chaleur}
	
	\newpage

Les échangeurs de chaleur sont des appareils permettant de transférer de la chaleur entre deux fluides à des températures différentes utilisés pour protéger les équipements des dommages thermiques ainsi que l’adaptation des matières à leurs températures adéquates de réaction, plusieurs systèmes sont conçus, dont on peut mentionner :

		\section{Groupe refroidissement YORK}
Qui est un échangeur de chaleur de type tubes et calandres non brassé installé à l’extérieur et reçoit le liquide chaud des bâtiments, passe dans les tubes en contact avec l’eau froide de la calandre, l’eau froide est reçue transporte par des pompes passe dans un radiateur pour abaisser de sa température et entre dans l’échangeur pour enlever la chaleur et est stockée ensuite dans les ballons pour refaire l’opération.

C’est le système de climatisation centralisé qui utilise un réfrigérant pour refroidir l’air dans le bâtiment.
 
\begin{figure}[h]
 \centering
 \includegraphics[scale=1]{../../SPECIALITE/stage/4/groupe york.jpg}
 \caption{Figure 8 Le groupe York
 }
 \end{figure}
  
 
		\section{Tour process refroidissement}
Un 2 -ème système a été opté par l’opérateur pour le refroidissement, 3 tours de process équipée chacune de 2 pompes centrifuges au sous-sol aspirant l’eau et le refoulant avec une pression élevée lui permettant de monter les tours. Les tours sont constituées d’un échangeur JACIR AIR TRAITEMENT à plaques à courants croisés non brassés
Des ventilateurs mécaniques prennent ensuite le relais pour refroidir l’eau froide évacuée de l’échangeur, la permettant de retourner dans l’échangeur et refaire l’opération.\\

\begin{tabular}{|c|c|}
\hline 
Capacité nominale	& 1279 kW   \\ 
\hline 
Débit	& 110 m³/h \\ 
\hline 
Surface d’échange	& 40.1 m² \\ 
\hline 
Pression maximale	& 6 bar\\ 
\hline 
Surpression maximale &	8 bar \\ 
\hline 
Température minimale	& 0 °C\\ 
\hline 
 Température maximale	& 125 °C   \\ 
\hline 
\end{tabular} 




		\section{Chaudière}
Un échangeur de chaleur de type … utilisé pour chauffer de la matière … un fluide jusqu’il atteigne la température de consigne pour l’utilisation industrielle

La chaudière utilise un liquide chauffé par la combustion du gaz naturel 
A tubes de fumée, ce type d’echangeur est composé d’une série de tubes disposés dans la partie superieure de la chaudière et qui sont exposés aux gaz de combustion chauds, le fluide circulant à l’interieur de eces tubes absorbe la chaleur de la combustion et se transforme en vapeur 

Tout d’abord le fluide est pompé dans la chaudière à travers les tubes conçs generalement pour maximiser la surface d’échange entre ce dernier et la source de chaleur, le bruleur de la chaudière est allumé utilisant le gaz naturel chauffant le fluide
La chaleur de la flamme est transférée à travers les tubes et au fluide qu’ils contiennent, ce transfer éléve la temperature du fluide , le fluide chauffé est pompé à travers les tubes de sortie de la chaudière et est acheminé vers le processus industriel qui necessite cette chaleur.
		\end{large}

\newpage

		\chapter{Conclusion }
		
		\newpage

        \begin{Large}
        En conclusion, j’ai pu, à travers de ce rapport de stage, présenter au lecteur le travail que j’ai effectué lors de mon stage. J’ai, dans un premier temps, présenté l’entreprise nationale de peinture, dans un second temps, le processus de production, ensuite les équipements qu’on a pu voir et enfin les différents systèmes d’échangeurs de chaleur utilisés.
        
      Cette brève visite de l’entreprise m’a permise aussi d’avoir une idée générale sur le fonctionnement de la société, le rôle d’un ingénieur dans cette dernière et les diverses tâches qui peuvent lui être attribuées, que l’erreur est impardonnable et la pertinence d’avoir des plans de secours.
      
      Ce fut une expérience intéressante qui a répondu à toutes mes questions et mes attentes et a confirmé mon choix de départ.
        \end{Large}

		


\end{document}
